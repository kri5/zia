\documentclass{report}
\usepackage{graphicx}
\usepackage{listings}
\usepackage{color}
\usepackage{fullpage}
\usepackage{geometry}
\usepackage[latin1]{inputenc}
\geometry{hmargin=2.5cm, vmargin=1.5cm}
\definecolor{grey}{rgb}{0.97,0.97,0.97}
\definecolor{orange}{rgb}{0.91,0.59,0.11}
\title{ZiAPI Specifications}
\author{\\beauze\_h\\courta\_c\\fauvet\_l\\boquet\_t}
\begin{document}
  \maketitle
  \tableofcontents
  \chapter{Chaine de Production}
    On peut decouper les grandes phases de fonctionnement du serveur en trois parties

    \begin{enumerate}
    \item L'initialistation du serveur
    \item La chaine de traitement des requetes
    \item L'arret du serveur
    \end{enumerate}

    \section{La chaine de production}
        Le traitement des requetes peut se scinder en 7 etapes distinctes, ce qui permettera aux modules de se greffer a 
        entre ces differentes etapes :
        \begin{enumerate}
            \item Connection d'un client, avant de recevoir des donnees de ce dernier
            \item Reception d'un header HTTP
            \item Reception d'un enventuel contenu apres le header
            \item Parsing de la requete
            \item Creation d'un header de reponse
            \item Envoi du corps de la reponse
            \item Fin du traitement de la requete
        \end{enumerate}

        Les modules pourront ainsi agir a differentes etapes de la chaine de production afin d'en modifier son comportement.
  \chapter{Outils Accessibles depuis les modules}
  \chapter{Description d'un module}
  \chapter{Types de Modules}
  \chapter{Example(s)}
    
\end{document}
