\chapter{Intégration des modules dans le serveur}
\section{Chargement des modules}
    Le chargement d'un module s'effectue suivant le shema suivant:
    \begin{enumerate}
        \item Lecture de l'emplacement du module depuis la configuration.
        \item Verification des symboles exportés.
        \item Identification des interfaces utilisées.
        \item Création et déploiement des instances correspondantes.
    \end{enumerate}
    
    \subsection{Configuration}
    Afin qu'un module soit dynamiquement chargé il doit être présent dans le fichier \textit{modules.conf} (inclus depuis \textit{zia.conf}). Il s'agit d'un arbre XML contenant les chemins vers les librairies dynamique (\textit{so} sous Linux ou \textit{dll} sous Windows).
    
    \subsection{Verification des symboles}
    Un certain nombre de symboles doivent être obligatoirement exportés afin de permettre un socle commun permettant l'intéraction avec le serveur:
    \begin{description}
        \item[create] permet de créer une nouvelle instance du module (plus precisement de sa classe principale).
        \item[destroy] détruie l'instance du module passée en paramètre.
        \item[name] retourne le nom du module.
    \end{description}
    Si l'un de ces symboles n'est pas correctement exporté le module ne pourra pas être chargé et une erreur sera retournée.

    \subsection{Identification des interfaces}
    L'API propose diverses interfaces pouvant être implémentées par les modules, ces interfaces contiennent diverses méthodes qui seront appelées à des points stratégiques du serveur. Elles sont toutes optionnelles et peuvent être héritées par les modules pour supporter tel ou tel évenement.


    Ces héritages sont donc utilisés conjointement avec des \textit{dynamic\_cast} pour déterminer quels évenements sont supportés par le module et ainsi l'inscrire aux différents points clés. 
